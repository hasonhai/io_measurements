\documentclass{acmsig}

\usepackage[color=yellow,obeyFinal]{todonotes}
\usepackage{graphicx}
\usepackage{epstopdf}
\usepackage{float}
\usepackage{varioref}
\usepackage{mathtools}
\usepackage{xspace}
\usepackage{subfigure}
\usepackage{hyperref}
\usepackage{balance}
\usepackage{listings}
\usepackage{algorithm}
\usepackage{algpseudocode}
\usepackage{paralist}
\usepackage{cite}
\usepackage{color}

\usepackage{xcolor}
\definecolor{dark-red}{rgb}{0.4,0.15,0.15}
\definecolor{dark-blue}{rgb}{0.15,0.15,0.4}
\definecolor{medium-blue}{rgb}{0,0,0.5}
\hypersetup{
  colorlinks, linkcolor={dark-red},
  citecolor={dark-blue}, urlcolor={medium-blue}
}

\newcommand{\etal}{\textit{et al.}\xspace}
\newcommand{\eg}{\textit{e.g.}\xspace}
\newcommand{\ie}{\textit{i.e.}\xspace}
\newcommand{\etc}{\textit{etc.}\xspace}
\newcommand{\vs}{\textit{vs.}\xspace}
\newcommand{\keyval}{$\langle\text{key}, \text{value}\rangle$\xspace}
\DeclarePairedDelimiter\floor{\lfloor}{\rfloor}


\newcommand{\noteby}[2]{\todo[inline]{#2\hspace*{\fill}\mbox{ --#1}}}

\lstset{frame=tb,
  language=SQL,
  aboveskip=3mm,
  belowskip=3mm,
  showstringspaces=false,
  columns=flexible,
  basicstyle={\small\ttfamily},
  numbers=none,
  numberstyle=\tiny\color{blue},
  stringstyle=\color{mauve},
  keywordstyle=\color{blue},
  commentstyle=\color{Brown},
  stringstyle=\color{mauve},
  breaklines=true,
  breakatwhitespace=true
  tabsize=3
}

% correct bad hyphenation here
\hyphenation{op-tical net-works semi-conduc-tor}


\begin{document}

% paper title
% can use linebreaks \\ within to get better formatting as desired
% \title{Automatic MapReduce ROLLUP}
\title{This is our title}


% author names and affiliations
% use a multiple column layout for up to three different
% affiliations
\numberofauthors{4}
\author{
\alignauthor
Ha Son Hai\\
       \affaddr{Orange/EURECOM}\\
       \email{sonhai@eurecom.fr}
\alignauthor
Daniele Venzano\\
       \affaddr{EURECOM}\\
       \email{venzano@eurecom.fr}
\and
\alignauthor 
Patrick Brown
       \affaddr{Orange}\\
       \email{p.brown@orange.fr}
\alignauthor 
Pietro Michiardi
       \affaddr{EURECOM}\\
       \email{michiard@eurecom.fr}
}

% make the title area
\maketitle


\begin{abstract}
Still very abstract
\end{abstract}

%%%%%%%%%%%%%%%%%%%%%%%%%%%%%%%%%%%%%%%%%%%%%%%%%%%%%%%%%%%%%%%%%%%%%
\section{Introduction}
%%%%%%%%%%%%%%%%%%%%%%%%%%%%%%%%%%%%%%%%%%%%%%%%%%%%%%%%%%%%%%%%%%%%%

%%%%%%%%%%%%%%%%%%%%%%%%%%%%%%%%%%%%%%%%%%%%%%%%%%%%%%%%%%%%%%%%%%%%%
\section{Background and Related Work}
%%%%%%%%%%%%%%%%%%%%%%%%%%%%%%%%%%%%%%%%%%%%%%%%%%%%%%%%%%%%%%%%%%%%%

%%%%%%%%%%%%%%%%%%%%%%%%%%%%%%%%%%%%%%%%%%%%%%%%%%%%%%%%%%%%%%%%%%%%%
\section{Problem statement}
\begin{itemize}
  \item \textit{What we do?}
  \begin{itemize}
    \item Experimental study of virtualization overheads for I/O operations
    \item We want to have the typical I/O ``pressure'' that comes from data-intensive applications
    \item We want to have the view from the ``guest operating system'' and the view from the applications (and see if there are differences)
  \end{itemize}
  \item \textit{Why and why is it important?}
  \begin{itemize}
    \item It is common wisdom that virtualization is ``bad'' for performance. Is it true?
    \item Given the knowledge about performance bottlenecks and eventual overheads of virtualization, is this going to be helpful for the applications we consider?
  \end{itemize}
\end{itemize}
%%%%%%%%%%%%%%%%%%%%%%%%%%%%%%%%%%%%%%%%%%%%%%%%%%%%%%%%%%%%%%%%%%%%%

%%%%%%%%%%%%%%%%%%%%%%%%%%%%%%%%%%%%%%%%%%%%%%%%%%%%%%%%%%%%%%%%%%%%%
\section{The methodology}

\subsection{The system under measurement}
\begin{itemize}
  \item Platform description
  \begin{itemize}
    \item Hardware
    \begin{itemize}
      \item Network architecture
      \item Disk subsystem: RAID controller, ...
    \end{itemize}
    \item Software
    \begin{itemize}
      \item Hypervisor
      \item OVS
      \item Host file-system
      \item LVM
    \end{itemize}
  \end{itemize}
\end{itemize}

\subsection{The measurement methodology}
\begin{itemize}
  \item A software framework to perform repeatable experiments
  \item A way to collect ``logs'' and analyze them
  \item A way to describe ``application-level'' I/O patterns, and implement them through measurements
  \item A way to instrument or to use the logs of data-intensive applications to collect measurement from their perspective
  \begin{itemize}
    \item Applications we consider: Hadoop, Spark, NoDB
  \end{itemize}
\end{itemize}

\subsection{The performance/overheads metrics}
\noteby{pm}{Should we consider over subscription as an important parameter??????}
\begin{itemize}
  \item Statistical methodology
  \item Metrics:
  \begin{itemize}
    \item Network
    \begin{itemize}
      \item Throughput
      \item Latency
      \item Jain Fairness index (in case of concurrent access patterns)
      \item CPU vs. network utilization
    \end{itemize}
    \item Disk
    \begin{itemize}
      \item Throughput
      \item Latency
      \item Jain Fairness index (in case of concurrent access patterns)
      \item CPU vs. disk utilization
    \end{itemize}
  \end{itemize}
\end{itemize}
%%%%%%%%%%%%%%%%%%%%%%%%%%%%%%%%%%%%%%%%%%%%%%%%%%%%%%%%%%%%%%%%%%%%%

%%%%%%%%%%%%%%%%%%%%%%%%%%%%%%%%%%%%%%%%%%%%%%%%%%%%%%%%%%%%%%%%%%%%%
\section{Results}

\begin{itemize}
  \item We present results per each scenario we consider, such that here we will have a number of subsections that are ``self-contained'', that is, they describe the scenario, to which application pattern it is matching, and then we present the results
\end{itemize}

\subsection{Single physical host}
\noteby{pm}{what does exactly mean ``fat''? should we make it with capacity equal to the sum of the capacity of multiple VMs?}
\noteby{dv}{We should take into account, for the measurements that include interference, which CPU Package the interfering vm is placed in}
\begin{itemize}
  \item Describe the ``scenario'' for this case (as a side note, we could also say we look at ``normal'' databases)
  \item Use the metrics above to compare bare-metal vs. virtual: this is essentially the overhead we want to study
  \item Disk
  \begin{itemize}
    \item Single (fat) VM
    \begin{itemize}
      \item Concurrency: from 1 to many parallel/concurrent I/O requests, using threads
      \item Interference: e.g. have a disturbing VM that uses a lot of CPU. Note that this makes much more sense if the host is ``oversubscribed''
      \item Single vs. multiple underlying disks
    \end{itemize}
    \item Multiple VMs
    \begin{itemize}
      \item Concurrency: from 1 to many parallel/concurrent I/O requests (equally distributed among VM)
      \item Interference: e.g. have a disturbing VM that uses a lot of CPU. Note that this makes much more sense if the host is ``oversubscribed''
      \item Single vs. multiple underlying disks
    \end{itemize}
  \end{itemize}
  \item Network
\end{itemize}



%%%%%%%%%%%%%%%%%%%%%%%%%%%%%%%%%%%%%%%%%%%%%%%%%%%%%%%%%%%%%%%%%%%%%





%%%%%%%%%%%%%%%%%%%%%%%%%%%%%%%%%%%%%%%%%%%%%%%%%%%%%%%%%%%%%%%%%%%%%
\section{Conclusion}
%%%%%%%%%%%%%%%%%%%%%%%%%%%%%%%%%%%%%%%%%%%%%%%%%%%%%%%%%%%%%%%%%%%%%



%%%%%%%%%%%%
% THIS PART IS FOR THE REFERENCE SECTION
%%%%%%%%%%%%
% \balance
% \bibliographystyle{abbrv}
% \bibliography{ref}

\end{document}
